\documentclass{article}
\usepackage{amsmath}
\usepackage{amsfonts}
\usepackage{amssymb}
\pagestyle{empty}
\usepackage{enumerate}
\usepackage{makeidx}
\usepackage{pgf,tikz}
\usepackage{tkz-euclide}
\usetkzobj{all}
\usepackage{graphicx}
\usepackage[margin = 0.5in]{geometry}
\raggedright
\author{Bryan Bain}
\begin{document}

Name \makebox[3in]{\hrulefill} \hfill Honors PreCalc P-Set

\subsubsection*{Vectors \hfill \makebox[0.35in]{\hrulefill} / 10}

Let \textbf{v} be the vector from initial point $P_1$ to terminal point $P_2$. Find the component form of $\vec{v}$, then find the magnitude.
    \begin{flalign*}
    1. \quad    &   P_1=(-4, \, -4), \, P_2=(6, \, 2)   &
    2. \quad    & P_1=(-8, \, 6), \, P_2=(-2, \, 3)   &
    3. \quad &  P_1=(-1, \, 7), \, P_2=(-7, \, -7)  &&\\[2.75in]
    \end{flalign*}
    
For the following, let $\textbf{u} = 2\textbf{i}-5\textbf{j}$, $\textbf{v} = -3\textbf{i}+7\textbf{j}$ and $\textbf{w}=-\textbf{i}-6\textbf{j}$. Find each specified vector or scalar.
    \begin{flalign*}
    4. \quad & \textbf{u}+\textbf{v}   &
    5. \quad & \textbf{u}-\textbf{v}   &
    6. \quad & \textbf{v}-\textbf{u}   &&\\[2.25in]
    7. \quad & -4\textbf{w}            &
    8. \quad & 3\textbf{v}-4\textbf{w} &
    9. \quad & ||2\textbf{u}||  &&\\
    \end{flalign*}
    
\newpage
    
Find the unit vector $\hat{v}$ that has the same direction as $\vec{v}$.
    \begin{flalign*}
    10. \quad &    \textbf{v} = 6\textbf{i}    &
    11. \quad &    \textbf{v} = 3\textbf{i}-2\textbf{j}    &
    12. \quad &    \textbf{v} = \textbf{i} + \textbf{j}    &&\\[3in]
    \end{flalign*}
    
Write the vector \textbf{v} in terms of \textbf{i} and \textbf{j} with the given magnitude and direction angle. Round to 2 decimal places for non-exact angles.
    \begin{flalign*}
    13. \quad   &    ||\textbf{v}|| = 6, \, \theta = 30^{\circ}  &
    14. \quad   &    ||\textbf{v}|| = 12, \, \theta = 225^{\circ}   &
    15. \quad   &    ||\textbf{v}|| = \frac{1}{2}, \, \theta = 113^{\circ}   &&\\
    \end{flalign*}
    



% Let \textbf{u} = $\langle a, \, b \rangle$ and \textbf{v} = $\langle c, \, d \rangle$. Prove each of the following properties.
% \begin{flalign*}
% 16. \quad   &   \textbf{u} + \textbf{v} = \textbf{v} + \textbf{u} &
% 17. \quad   &   k(\textbf{u} + \textbf{v}) = k\textbf{u} + k\textbf{v} &
% 18. \quad   &   |k\textbf{u}| = |k|\cdot |\textbf{u}|    &&\\[2in]
% \end{flalign*}




% \fbox{\emph{Level 3}}
% \newline\\


% An object is said to be in \emph{equilibrium} if the sum of the net forces acting upon it is 0. 
% \newline\\

% In the figure, the weight $w$ hangs from two cables. Find an expression for the tension on each cable if the weight is in equilibrium (don't forget gravity's effect on the weight).


% \begin{center}
% \begin{tikzpicture}
%     \draw [fill = gray] (0,0) rectangle (5,-0.25);
%     \draw (0,-0.25) -- (3.5,-2);
%     \draw (5,-0.25) -- (3.5,-2);
%     \draw (3.5,-2) -- (3.5,-2.5);
%     \draw (2.5,-2.5) rectangle (4.5,-3.5);
%     \node at (3.5,-3) {$w$};
%     \node at (0.75,-0.2) [anchor = north west] {$\theta_1$};
%     \node at (4.75,-0.2) [anchor = north east] {$\theta_2$};
%     \node at (2,-1.15) [anchor = north east] {$T_1$};
%     \node at (4.25,-1.15) [anchor = north west] {$T_2$};
% \end{tikzpicture}
% \end{center}





\newpage

\begin{enumerate}
\item $10\textbf{i}+6\textbf{j}; \, \sqrt{136}$    
\item $6\textbf{i}-3\textbf{j}; \, \sqrt{45}$       
\item $-6\textbf{i}-14\textbf{j}; \, \sqrt{232}$    
\item $-\textbf{i}+2\textbf{j}$     
\item $5\textbf{i}-12\textbf{j}$    
\item $-5\textbf{i}+12\textbf{j}$   
\item $4\textbf{i}+24\textbf{j}$    
\item $-5\textbf{i}+45\textbf{j}$  
\item $\sqrt{116}$                  
\item $\textbf{i}$                  
\item $\frac{3\sqrt{13}}{13}\textbf{i}-\frac{2\sqrt{13}}{13}\textbf{j}$    
\item $\frac{\sqrt{2}}{2}\textbf{i} + \frac{\sqrt{2}}{2}\textbf{j}$    
\item $3\sqrt{3}\textbf{i} + 3\textbf{j}$
\item $-6\sqrt{2}\textbf{i}-6\sqrt{2}\textbf{j}$    
\item $\approx -0.20\textbf{i}+0.46\textbf{j}$   
\end{enumerate}


\end{document}
