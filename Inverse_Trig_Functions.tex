\documentclass{article}
\usepackage[latin1]{inputenc}
\usepackage[english]{babel}
\usepackage{amsmath}
\usepackage{amsfonts}
\usepackage{amssymb}
\usepackage{makeidx}
\usepackage{xcolor}
\usepackage{systeme}
\usepackage{enumerate}
\pagestyle{empty}
\usepackage{pgf,tikz}
\usepackage{tkz-euclide}
\usetkzobj{all}
\raggedright
\usetikzlibrary{arrows}
\usepackage{graphicx}
\usepackage[margin = 0.5in]{geometry}
\begin{document}

Name \makebox[3in]{\hrulefill} \hfill Hon. PreCalc P-Set

\subsubsection*{Inverse Trig Functions \hfill \makebox[0.35in]{\hrulefill} / 10}


% \fbox{\emph{Level 1}}
% \newline\\

Find the exact value of each. 
\begin{flalign*}
1.  \quad   &   \cos^{-1} \frac{\sqrt{3}}{2}  &
2.  \quad   &   \tan^{-1} 1     &
3.  \quad   &   \arcsin(-0.5)       &&\\[0.5in]
4.  \quad   &   \sec^{-1}\sqrt{2}   &
5.  \quad   &   \cot^{-1}(-\sqrt{3}) &
6.  \quad   &   \sin^{-1} \frac{-\sqrt{2}}{2}   &&\\[0.5in]
7.  \quad   &   \csc^{-1}( -2)    &
8.  \quad   &   \cos^{-1} 0     &
9.  \quad   &   \cos^{-1}\left(\cos\left(\frac{4\pi}{3}\right)\right) &&\\[1in]
10. \quad   &   \sin\left(\cos^{-1}\left(\frac{4}{5}\right)\right)  &
11. \quad   &   \sec\left(\tan^{-1}\left(\frac{\sqrt{3}}{3}\right)\right)   &
12. \quad   &   \sin(\arctan(x))    &&\\[1.5in]
\end{flalign*}

% \fbox{\emph{Level 2}}
% \newline\\

Write the expression as a single algebraic expression.
\begin{flalign*}
13. \quad   &   \sin\left(\tan^{-1}\frac{x}{\sqrt{5}}\right)    &
14. \quad   &   \tan\left(\sec^{-1}\left(\frac{\sqrt{x^2+25}}{x}\right)\right)  &
15. \quad   &   \sin\left(\arcsin\frac{x}{\sqrt{2}}\right) &&\\[1.5in]
\end{flalign*}

% \fbox{\emph{Level 3}}
% \newline\\

Show that the following are true for $\csc^{-1}x$, $\sec^{-1}x$ and $\cot^{-1}x$.
\begin{flalign*}
16. \quad   &   \csc^{-1}x = \sin^{-1}\left(\frac{1}{x}\right) \text{for } x\geq 1   &
17. \quad   &   \sec^{-1}x = \cos^{-1}\left(\frac{1}{x}\right) \text{for } x \geq 1   &
18. \quad   &   \cot^{-1}x = \tan^{-1}\left(\frac{1}{x}\right) \text{for } x > 0   &
\end{flalign*}


\newpage


\textbf{Inverse Trig Functions KEY}

\begin{enumerate}
    \item $\frac{\pi}{6}$
    \item $\frac{\pi}{4}$
    \item $-\frac{\pi}{6}$
    \item $\frac{\pi}{4}$
    \item $-\frac{\pi}{6}$
    \item $-\frac{\pi}{4}$
    \item $-\frac{\pi}{6}$
    \item $\frac{\pi}{2}$
    \item $\frac{2\pi}{3}$
    \item $\frac{3}{5}$
    \item $\frac{2\sqrt{3}}{3}$
    \item $\dfrac{x}{\sqrt{x^2+1}} = \dfrac{x\sqrt{x^2+1}}{x^2+1}$
\end{enumerate}



\end{document}
