\documentclass[11pt,a4paper]{exam}
\usepackage[latin1]{inputenc}
\usepackage[english]{babel}
\usepackage{amsmath}
\usepackage{amsfonts}
\usepackage{amssymb}
\usepackage{makeidx}
\usepackage{xcolor}
\usepackage{systeme}
\usepackage{enumerate}
\usepackage{pgf,tikz}
\raggedright
\pagestyle{empty}
\usetikzlibrary{arrows,calc}
\usepackage{graphicx}
\usepackage[margin = 0.5in]{geometry}
\begin{document}

%\iffalse

Name \makebox[2.5in]{\hrulefill} \hfill Honors PreCalc P-Set 

\subsubsection*{Logarithmic Functions \hfill \makebox[0.35in]{\hrulefill} / 10}  
Write each in either exponential or logarithmic form.
\begin{flalign*}
1. \quad    &   \log_5{125}=3             &
2. \quad    &   \log_{81}3=\frac{1}{4}    &
3. \quad    &   \log{10,000}=4            &&\\[0.5in]
4. \quad    &   2^6 = 64                  &
5. \quad    &   10^{-5}=0.00001           &
6. \quad    &   \left(\dfrac{1}{27}\right)^{-1/3}=3   &&\\[0.5in]
\end{flalign*}


Evaluate each of the following. Round to 4 decimal places when necessary.
\begin{flalign*}
7. \quad    &   \log_3{40}            &
8. \quad    &   \log_{50}{8,675,309}    &
9. \quad    &   \ln{380}        &&\\[0.5in]
\end{flalign*}


State the domain of each.
\begin{flalign*}
10. \quad   &   \log_2{(x+5)}               &
11. \quad   &   \log\left(|x-1|\right)    &&\\[1.25in]
12. \quad   &   \log{\left(x^2+5x+6\right)}   &
13. \quad   &   \ln\left(\frac{x+2}{x^2-1}\right)            &&\\[1.5in]
\end{flalign*}


Describe the transformation done to $y=\log x$ to produce each. Be specific.
\begin{flalign*}
14. \quad   &   f(x)=5\log{(x+2)}-8             &
15. \quad   &   g(x)=-\frac{1}{2}\log{(x-3)}+1  &
16. \quad   &   h(x)=\log\left(-3x+12\right)     &&\\
\end{flalign*}

\newpage


% \fbox{\emph{Level 2}}
% \newline\\

% Find $f^{-1}$ for each. Check by graphing $f$, $f^{-1}$ and $y=x$ on the same plane.
% \begin{flalign*}
% 15. \quad   &   f(x) = 3\ln(x+1)  &
% 16. \quad   &   f(x) = -2\log (x) - 3 &
% 17. \quad   &   f(x) = \ln(x - 5)   &&\\[3in]
% \end{flalign*}



% \fbox{\emph{Level 3}}
% \newline\\

% In calculus, Taylor Polynomials can be used to approximate the function $f(x) = \ln x$ for values of $x$.
% \\
% This is also how your calculator finds the value of a natural logarithm. 
% \newline\\

% The Taylor series for $f(x) = \ln x$ is 
% \[
%     \ln x = (x-1) - \frac{(x-1)^2}{2} + \frac{(x-1)^3}{3} - \frac{(x-1)^4}{4} + \frac{(x-1)^5}{5} - \cdots
% \]

% Calculate each of the following using the Taylor series given above. Round your answers to 4 decimal places.
% \begin{flalign*}
% 18. \quad   &   \ln 5   &
% 19. \quad   &   \ln 1   &
% 20. \quad   &   \ln e   &&\\
% \end{flalign*}


\newpage

%\fi

\textbf{Logarithmic Functions KEY}
\begin{enumerate}
    \item $5^3=125$
    \item $81^{\frac{1}{4}}=3$
    \item $10^4=10,000$
    \item $\log_2{64}=6$
    \item $\log{0.00001}=-5$
    \item $\log_{\frac{1}{27}}3=-\frac{1}{3}$
    \item 3.3578
    \item 4.0838
    \item 5.9402
    \item $(-5, \, \infty)$
    \item $(-\infty, \, 1)\cup(1, \, \infty)$
    \item $(-\infty, \, -3)\cup(-2, \, \infty)$
    \item $(-2,-1) \cup (1,\infty)$
    \item V. Stretch by factor of 5, 2 units left, 8 units down
    \item Reflect across $x$-axis, V. Compression by factor of 2, 3 units right, 1 unit up
    \item Shift left 12 units, horizontal compression by factor of 3, reflect across $y$-axis
\end{enumerate}




\end{document}
