\documentclass[11pt,a4paper]{exam}
\usepackage[latin1]{inputenc}
\usepackage[english]{babel}
\usepackage{amsmath}
\usepackage{amsfonts}
\usepackage{amssymb}
\usepackage{makeidx}
\usepackage{xcolor}
\usepackage{systeme}
\usepackage{enumerate}
\usepackage{pgf,tikz}
\raggedright
\pagestyle{empty}
\usetikzlibrary{arrows,calc}
\usepackage{graphicx}
\usepackage[margin = 0.5in]{geometry}
\begin{document}

%\iffalse

Name \makebox[2.5in]{\hrulefill} \hfill Honors PreCalc P-Set 

\subsubsection*{Polynomial and Rational Inequalities   \hfill \makebox[0.35in]{\hrulefill} / 10} 

Solve each. Write your answers using interval notation.
\begin{flalign*}
1.  \quad   &   x^3-15x^2+75x-125 \leq 0    &
2.  \quad   &   2x^3-19x^2+55x-50 > 0       &&\\[2in]
3.  \quad   &   9x^4+6x^3-284x^2+730x-525 \leq 0    &
4.  \quad   &   -3x^3-19x^2-5x+75 \leq 0    &&\\[2in]
5.  \quad   &   \dfrac{2x+10}{2x-7} \leq 0  &
6.  \quad   &   \dfrac{x+5}{x-3} \leq 0     &&\\[2in]
7.  \quad   &   \dfrac{x+7}{x-3} \geq 0     &
8.  \quad   &   \dfrac{x-1}{x+2} \geq 0     &&\\
\end{flalign*}

\newpage

\begin{flalign*}
9.  \quad   &   \dfrac{-x+43}{x+5} \leq 3   &
10. \quad   &   \dfrac{-x+254}{5x-35} > -4  &&\\[2.75in]
11. \quad   &   \dfrac{2x^2-17x+30}{2x-2} \geq 0    &
12. \quad   &   \dfrac{x+6}{x^2-16} \geq 0 &&\\[2.75in]
13. \quad   &   \dfrac{2x^2+2x-4}{2x-1} > 0 &
14. \quad   &   \dfrac{3x-3}{3x^2-13x+4} < 0    &&\\
\end{flalign*}


% \fbox{\emph{Level 2}}
% \newline\\


% A dog lover builds a rectangular dog play area with 100 ft of fencing and an area of at least 600 sq. ft. The outer wall of the house will be used as the fourth side.
% \newline\\


% 15. Write an inequality that could be used to find the possible lengths of the sides. \vspace{0.75in}


% 16. Solve the inequality to find the possible lengths of fencing to be used. Round to 1 decimal place if necessary.


% \newpage



% \fbox{\emph{Level 3}}
% \newline\\


% When analyzing the growth of functions (in this case, polynomials), mathematicians and computer scientists use big-\textit{O} notation. 
% \newline\\


% Big-\textit{O} notation gives us an idea for function growth and computer run-time for algorithms.
% \newline\\


% $f(x)$ is said to be $O(g(x))$ if there are constants $C$ and $k$ such that 
% \[
% |f(x)| \leq C\cdot|g(x)|
% \]

% whenever $x \geq k$. 
% \newline\\


% 17. Show that $f(x)=5x^3 + 2x^2 + 3x + 5$ is $O(x^3)$.  
% \newline\\

% \emph{Hint}: Use the facts that $2x^2 \leq x^3$, $3x \leq x^3$, and $5 \leq x^3$ when $x\geq2$ to help you find the value of $C$.





\newpage


\textbf{Polynomial and Rational Inequalities KEY}
\begin{enumerate}
    \item $(-\infty, \ 5]$
    \item $\left(2, \ \frac{5}{2}\right) \cup (5, \ \infty)$
    \item $[-7, \ 3]$
    \item $[-5, \ -3] \cup \left[\frac{5}{3}, \ \infty\right)$
    \item $\left[-5, \ \frac{7}{2}\right)$
    \item $[-5, \ 3)$
    \item $(-\infty, \ -7] \cup (3, \ \infty)$
    \item $(-\infty, \ -2) \cup [1, \ \infty)$
    \item $(-\infty, \ -5) \cup [7, \ \infty)$
    \item $(-\infty, \ -6) \cup (7, \ \infty)$
    \item $\left(1, \ \frac{5}{2}\right] \cup [6, \ \infty)$
    \item $(-6, \ -4) \cup (4, \ \infty)$
    \item $\left(-2, \ \frac{1}{2}\right) \cup (1, \ \infty)$
    \item $\left(-\infty, \ \frac{1}{3}\right) \cup (1, \ 4)$
\end{enumerate}






\end{document}
